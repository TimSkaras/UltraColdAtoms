\documentclass[onecolumn,english,aps,pra]{revtex4}
\usepackage{amssymb}
\usepackage{amsmath}
\usepackage{graphicx}
\usepackage{epstopdf}
\usepackage{bm}
\usepackage{braket}
\usepackage{color}
\usepackage{rotating,booktabs}
\usepackage{array}


\begin{document}
\title{Two hard-core anyons in harmonic trap}
%\author{Li Yang$^1$, and Han Pu$^{1,2}$}

%\affiliation{$^{1}$Department of Physics and Astronomy, and Rice Center for Quantum Materials,
%Rice University, Houston, TX 77251, USA \\
%$^2$Center for Cold Atom Physics, Chinese Academy of Sciences, Wuhan 430071, P. R. China}

\maketitle
 
\section{general formulation}
A two spinless particle in 1D are described by the two-body wavefunction $\Psi(x_1,x_2)=\langle x_1,x_2|\Psi \rangle$. The reduced one-body density operator is \[ \hat{\rho}_1 = \int dx_2\, \langle x_2| \Psi \rangle \langle \Psi |x_2 \rangle = \int dx_1 \int dx'_1 \int dx_2\, |x'_1\rangle \langle x'_1| \langle x_2| \Psi \rangle \langle \Psi |x_2 \rangle  |x_1\rangle \langle x_1| = \int dx_1 \int dx'_1 \, |x'_1\rangle \rho(x_1,x_1') \langle x_1|   \]  where the one-body density matrix (OBDM) is defined as:
\[ \rho(x,x')= \langle x'|\hat{\rho}_1| x \rangle = \int dx_2 \, \Psi^*(x,x_2) \Psi(x',x_2) \]
The real space and momentum space density profiles can be easily obtained from OBDM as
\begin{eqnarray}
	n(x) &=& N \langle x| \hat{\rho}_1 |x \rangle = N \rho(x,x) = N \int dx_2 \, \Psi^*(x,x_2) \Psi(x,x_2) = N \int dx_2 \, |\Psi(x,x_2)|^2   \\
	n(p) &=& N \langle p| \hat{\rho}_1 |p \rangle =\frac{N}{2\pi} \int dx \int dx' \,e^{ip(x-x')}\, \rho(x,x')
	\end{eqnarray}
with $N=2$. Furthermore, given a single-particle state $\varphi(x)$, the probability to find the particle in this state is \[  P_\varphi = \langle \varphi| \hat{\rho}_1 |\varphi \rangle= \int dx \int dx' \,\varphi(x) \varphi^*(x') \rho(x,x') \] 	

We will focus on the ground state of such a two-particle system in a harmonic trap with hardcore interaction. The Hamiltonian of the system reads:
\[ H =H_0+H_{\rm int}= -\frac{\hbar^2}{2m} \left( \partial_1^2 + \partial_2^2 \right) + \frac{1}{2} m\omega^2 (x_1^2+x_2^2) + g\delta(x_1-x_2)  \] where the interaction strength takes the limit $g \longrightarrow \infty$.

\section{free fermions in harmonic trap}
Now consider two free fermions in a harmonic trap, whose single-particle eigenstates are labeled as $\varphi_i(x)$ where $i =0,$ 1, 2, ..., and satisfy the orthonormal condition $ \int dx \, \varphi_i^*(x) \varphi_j(x) = \delta_{ij}$. The ground state wavefunction of the two-fermion system is given by \[ \Psi_F(x_1,x_2) = \frac{1}{\sqrt{2}} \,\left[ \varphi_0(x_1) \varphi_1(x_2) -\varphi_0(x_2) \varphi_1(x_1) \right]  \]
Using the expression above, one can readily find the following:
\begin{eqnarray}
\rho_F(x,x') &=& \frac{1}{2} \left[ \varphi_0^*(x) \varphi_0(x') +  \varphi_1^*(x) \varphi_1(x') \right] \\
n_F(x) &=& |\varphi_0(x)|^2 + |\varphi_1(x)|^2 \\
n_F(p) &=& |\tilde{\varphi}_0(p)|^2 + |\tilde{\varphi}_1(p)|^2 
\end{eqnarray}
where \[ \tilde{\varphi}_i(p) = \frac{1}{\sqrt{2\pi}} \int dx \, e^{-ipx}\,\varphi_i(x) \] is the single-particle momentum space wavefunction.

Finally, the probability of finding the particle in the $i^{\rm th}$ harmonic oscillator eigenstate $\varphi_i(x)$ is given by \[ P_0=P_1=1/2\,,\;\;\;P_{i\neq 0,1}=0 \]
The corresponding reduced density operator can be written as 
\[ \hat{\rho}_F = \frac{1}{2} \left( |\varphi_0 \rangle \langle \varphi_0| + |\varphi_1 \rangle \langle \varphi_1| \right) \] with the associated von Neumann entropy \[ S = {\rm Tr}[-\hat{\rho}_F \ln \hat{\rho}_F ] =\ln 2 \]

\section{hardcore anyons in harmonic trap}
Now consider two hardcore anyons in a harmonic trap. Using the anyon-Fermi mapping, the corresponding wavefunction of the anyons is given by 
\[ \Psi_\kappa (x_1,x_2) = A_\kappa(x_2-x_1) \,\Psi_F(x_1,x_2) \]
where $\kappa$ is the anyon statistical parameter and 
\[ A_\kappa (x_2-x_1) = e^{i\pi (1-\kappa) \theta(x_2-x_1)}= \left\{ \begin{array}{ll} e^{i\pi(1-\kappa)}\,, & x_2>x_1 \\ 1 \,, & x_2 <x_1 \end{array}  \right.  \] where $\theta(x_2-x_1)$ is the Heaviside step function.
The anyon wavefunction satisfies the following exchange properties:
\[ \Psi_\kappa(x_1,x_2) = e^{i\pi \kappa \epsilon(x_1-x_2)} \,\Psi_\kappa(x_2,x_1) \]
where \[ \epsilon(x) =  \left\{  \begin{array}{cl} 1\,, & x>0 \\ 0 \,, & x=0 \\ -1\,, & x<0 \end{array} \right.\]
Two special cases are: (1) For $\kappa=0$, $ A_\kappa (x_2-x_1) = {\rm sgn}(x_1-x_2)$, and the anyons correspond to hardcore bosons; (2) for $\kappa=1$, $A_\kappa (x_2-x_1) = 1$, and the anyons correspond to hardcore, i.e., free, fermions. Note that $A_\kappa = A^{\kappa+2}$, hence we can restrict the values of $\kappa$ to be $\kappa \in [0, 2)$. 

Since $|\Psi_\kappa| = |\Psi_F|$ independent of $\kappa$, the real space density profile $n(x)$ is independent of $\kappa$. In other words, all hardcore anyons have the same real space density profile independent of their statistical parameter.

From the two-body wave function for two hard-core anyons in a harmonic trap, the OBDM can be calculated using the formula given in the first section combined with the anyon-fermi mapping:
\begin{align*}
\rho_\kappa(x,x')= \langle x'|\hat{\rho}_1| x \rangle &= \int dx_2 \, \Psi_{\kappa}^*(x,x_2) \Psi_\kappa(x',x_2)\\
& = \int dx_2 \,  e^{-i\pi (1-\kappa) \theta(x - x_2)} e^{i\pi (1-\kappa) \theta(x' - x_2)} \Psi_{F}^*(x,x_2) \Psi_{F}(x',x_2)\\
& = \rho_{F}(x,x') + \epsilon(x' - x)(e^{\epsilon(x' - x) i\pi (1-\kappa)} - 1)\int_{x}^{x'} dx_2 \, \Psi_{F}^*(x,x_2) \Psi_{F}(x',x_2)
\end{align*}

The latter integral can be calculated ``analytically" using the error function. 

Using the projection formula given in the first section, the anyon OBDM can be used to numerically calculate the anyon's projection values. For one of the fermions in the two-body ground state, the probability of finding the particle in either the ground state or the first excited state was $P_{0} = P_{1} = \frac{1}{2}$ and all other projections were 0. An anyon in the two-body ground state can be found in any of the excited states (assuming $\kappa \neq 1$), however, with diminishing probability for excited states with more energy. 

For instance, if two hard-core bosons are in the two-body ground state of a harmonic trap, then the probability of finding one of those particles in the $\phi_0$ state is .718, the probability for the $\phi_1$ state is .1667, and so on, as is illustrated in the top left plot of Fig. \ref{fig:projections} ($\kappa = 0$).
\begin{figure}[h]
	\includegraphics[scale=.5]{"../Plots/Anyon Projection Values"}
	\caption{Projections}
	\label{fig:projections}
\end{figure}
\begin{figure}[h]
\includegraphics[scale=.5]{"../Plots/Momentum"}
\caption{Momentum Distribution}
\label{fig:momentum}
\end{figure}

The OBDM can also be used to numerically find the momentum distribution of an anyon for various values of $\kappa$ (see Fig. \ref{fig:momentum}). The y-axis indicates the probability of finding the anyon with a particular momentum. Only the hard-core boson case ($\kappa = 0, 2$) and the fermion case ($\kappa = 1$) have momentum distributions that are symmetric about $p = 0$.

The von Neumann entropy can be determined from the OBDM by transforming from the position basis to the harmonic potential eigenbasis. The resulting matrix will be infinite in size, but for the purposes of calculations only the first five terms in each dimension need to be considered -- projections after $P_5$ are on the order of $10^{-3}$ and are thus negligible. If $\lambda_1, \lambda_2,\ldots$ are the eigenvalues of this matrix, the von Neumann entropy is given by
\[ S =  -\sum_i^\infty \lambda_i \log(\lambda_i) \]
Thus, the von Neumann entropy for hard-core anyons in the two-body ground state of the harmonic oscillator are given in Fig. \ref{fig:entropy} for various values of $\kappa$. The fermion case ($\kappa = 1$) obtains the analytic result $S = \log(2) = 0.6931$. 

\begin{figure}[h]
\includegraphics[scale=.5]{"../Plots/EntropyPlot"}
\caption{Plot of von Neumann Entropy $(S)$}
\label{fig:entropy}
\end{figure}

Evidently, the entropy is greatest for $\kappa$ near 1 but not 1 and obtains its minimum value for hard-core bosons ($\kappa = 0$). It is also reassuring that the von Neumann entropy is symmetric about $\kappa = 1$, so $S_{\kappa = 1/2} = S_{\kappa = 3/2}$ and so on.

A. Minguzzi et al.\footnotemark\,claim that for large $p$ the momentum distribution $n(p)$ of hardcore bosons decays like $1/p^4$. 
\footnotetext{A. Minguzzi, P. Vignolo, and M. Tosi, Physics Letters A \textbf{294}, 222 (2002).}
I have attempted to reproduce this calculation below in Section IV. Their analytic result is that
\[ \lim_{p \rightarrow \infty} n(p) = 2\sqrt{\frac{2}{\pi }} \frac{(\hbar m \omega)^{3/2}}{p^4}  \]
In fig. \ref{fig:tails} I have plotted on a log-log scale the anyonic momentum distribution $n_\kappa (p)$ for various $\kappa$. The orange line depicts the value $\log(n_\kappa(p))$ for $p \in [-25, -2] \cup [2,25]$. The blue line is the log-log plot of $1/p^4$. These plots confirm that hardcore bosons as well as anyons follow a $1/p^4$ momentum distribution decay. Even though anyons do not have a symmetric momentum distribution for $\kappa \in (0,1) \cup (1,2)$ (see fig. \ref{fig:momentum}), it is still true that the anyonic momentum tails follow a $1/p^4$ decay, regardless of the sign of $p$ (see fig. \ref{fig:tails}).

However, these numerical results are not in complete accord with the analytic coefficient presented by Minguzzi et al. For instance, if $N(p)$ is the numerically calculated value for the momentum distribution at momentum $p$, then
\[ N(p) = \dfrac{C}{p^4} \]
should hold for some constant $C$ assuming $p$ is sufficiently large. This constant, therefore, can be approximated at each point as $C = N(p) * p^4$. If $N(p)$ has been calculated at values $\{ p_{1},p_{2}, \ldots, p_{k} \}$, then $C \approx \text{Mean}(N(p_{1}) * p_{1}^4, N(p_{2}) * p_{2}^4, \ldots, N(p_{k}) * p_{k}^4) $. Using this fact, one can infer from the data above that $C \approx 0.513959 $. 

If the $\frac{1}{2\pi}$ prefactor is included in Minguzzi's coefficient, my numerical results should be in agreement with Minguzzi's coefficient, yet they still differ by a factor of 2. According to Minguzzi's paper, when the prefactor is included we have $C = \frac{1}{\pi} \sqrt{\frac{2}{\pi}} \approx 0.253974\ldots$, which is about half what I calculated from my numerical data.

\section{momentum tail coefficient for hcb}

In the following section, I will try to reproduce the result from Minguzzi's paper that 
\[ \lim_{p \rightarrow \infty} n(p) = 2\sqrt{\frac{2}{\pi }} \frac{(\hbar m \omega)^{3/2}}{p^4}  \]

\begin{center}
\begin{figure}[h]
	\includegraphics[scale=.45]{"../Plots/FullMomentumTails"}
	\caption{Momentum Tails for Anyons}
	\label{fig:tails}
\end{figure}
\end{center}

Recall that two body ground state for two fermions in a harmonic trap is
\[ \Psi_F(x_1,x_2) = \frac{1}{\sqrt{2}} \,\left[ \varphi_0(x_1) \varphi_1(x_2) -\varphi_0(x_2) \varphi_1(x_1) \right]  \]
\[ \varphi_{n}(x) = \dfrac{1}{\pi^{1/4} \sqrt{\alpha 2^n n!}} H_{n}(x/\alpha) e^{-\frac{1}{2}(x/\alpha)^2} \]
where $\alpha = \sqrt{\frac{\hbar}{m \omega}}$. Simplifying, 
\[ \Psi_{F}(x_{1},x_{2}) = \dfrac{1}{\sqrt{\pi} \alpha^2} e^{-\frac{1}{2\alpha^2} (x_{1}^2 + x_{2}^2)} (x_{2} - x_{1}) \]
By the Bose-Fermi mapping, the two-body ground state for two hard-core bosons must be 
\[ \Psi_{B}(x_{1},x_{2}) = \dfrac{1}{\sqrt{\pi} \alpha^2} e^{-\frac{1}{2\alpha^2} (x_{1}^2 + x_{2}^2)} |x_{2} - x_{1}|  \]
The one-body density matrix is then
\begin{align*}
\rho_{B}(x, x') &= \int_{-\infty}^{\infty} \Psi_{B}^{*}(x ,x_{2}) \Psi_{B}(x' ,x_{2}) dx_{2}\\
& = \dfrac{1}{\alpha^4 \pi} e^{-\frac{1}{2\alpha^2} (x^2 + x'^2)} \int_{-\infty}^{\infty} e^{-x_{2}^{2}/\alpha^2} |x_{2} - x| |x_{2} - x'| dx_{2}
\end{align*}
Let $Q_{2} = x_{2}/\alpha$, $Q = x/\alpha$, and $Q' = x'/\alpha$ to obtain
\begin{equation}
\rho_{B}(x, x') = \dfrac{1}{\alpha \pi} e^{-\frac{1}{2} (Q^2 + Q'^2)} \int_{-\infty}^{\infty} e^{-Q_{2}^{2}} |Q_{2} - Q| |Q_{2} - Q'| dQ_{2}
\end{equation}
Please note that I have only scaled $Q_{2}$ via u-substitution. The other variables have just been rewritten to simplify the expression. I am not scaling the length. Variables $Q$ and $Q'$ are just shorthand for $ x/\alpha$ and $x'/\alpha$ respectively.

The momentum distribution is
\begin{align*}
n_{B}(p) &= \dfrac{N}{2 \pi \hbar} \int_{-\infty}^{\infty} dx \int_{-\infty}^{\infty} dx' e^{i \frac{p}{\hbar} (x - x')} \rho_{B}(x,x')\\
& = \dfrac{N}{2 \alpha \pi^2 \hbar} \int_{-\infty}^{\infty} \alpha dQ \int_{-\infty}^{\infty} \alpha dQ' e^{i \alpha \frac{p}{\hbar} (Q - Q')}  e^{-\frac{1}{2} (Q^2 + Q'^2)} \int_{-\infty}^{\infty} e^{-Q_{2}^{2}} |Q_{2} - Q| |Q_{2} - Q'| dQ_{2}\\
& = \dfrac{N\alpha }{2 \pi^2 \hbar} \int_{-\infty}^{\infty} dQ \int_{-\infty}^{\infty} dQ' e^{i \alpha \frac{p}{\hbar} (Q - Q')}  e^{-\frac{1}{2} (Q^2 + Q'^2)} \int_{-\infty}^{\infty} e^{-Q_{2}^{2}} |Q_{2} - Q| |Q_{2} - Q'| dQ_{2}\\
& =  \dfrac{N\alpha }{2 \pi^2 \hbar} \int_{-\infty}^{\infty}  dQ_{2} e^{-Q_{2}^2} \int_{-\infty}^{\infty} dQ  e^{i \alpha \frac{p}{\hbar} Q} e^{-\frac{1}{2}Q^2} |Q_{2} - Q|  \int_{-\infty}^{\infty} dQ'  e^{-i \alpha \frac{p}{\hbar} Q'} e^{-\frac{1}{2}Q'^2} |Q_{2} - Q'| \\
& = \dfrac{N\alpha }{2 \pi^2 \hbar} \int_{-\infty}^{\infty}  dQ_{2} e^{-Q_{2}^2} \left| \int_{-\infty}^{\infty} dQ  e^{i \alpha \frac{p}{\hbar} Q} e^{-\frac{1}{2}Q^2} |Q_{2} - Q| \right|^2
\end{align*}

The integral over $Q$ can be evaluated asymptotically as

\begin{equation}
\int_{-\infty}^{\infty} dQ  e^{i \alpha \frac{p}{\hbar} Q} e^{-\frac{1}{2}Q^2} |Q_{2} - Q| = \dfrac{-2 \hbar ^2 e^{-\frac{1}{2} Q_{2}^2}}{\alpha^2 p^2}
\end{equation}

for large p. Thus,
\begin{align}
\lim_{p \rightarrow \infty} n_{B}(p) & = 
\dfrac{2}{\pi} \sqrt{\dfrac{2}{\pi}} \dfrac{(\hbar m \omega)^{3/2}}{p^4}
\end{align}
%
from which we can infer that the Tan contact for two harmonically trapped hard-core bosons is
\[ 
C_{B}= \dfrac{2}{\pi} \sqrt{\dfrac{2}{\pi}}
\]
This is consistent with the ground state energy of two bosons in the TG ($g \rightarrow \infty$) limit, which reads
\[ 
E_{B} = E_{0} - 2 \sqrt{\dfrac{2}{\pi}}\dfrac{1}{g}
\]
and from the adiabatic sweep theorem, we have (see Eur. Phys. J. Special Topics \textbf{226}, 1583 (2017))
\[
\dfrac{dE_{B}}{d(1/g)} = -2\sqrt{\dfrac{2}{\pi}} = - \pi C_{B}
\]

\section{momentum tail coefficient for hca}

To find the momentum tail coefficient for hard-core anyons, the calculation is very similar to the calculation for hard-core bosons. I will give the derivation for the result

\[
\lim_{p \rightarrow \infty} n_\kappa(p) = \cos^2\left(\frac{\pi \kappa}{2}\right) \dfrac{2}{\pi} \sqrt{\dfrac{2}{\pi}} \dfrac{(\hbar m \omega)^{3/2}}{p^4}
\]
where the $\kappa$ subscript denotes the anyon parameter of the anyon two-body system. 

Like before the fermion and hard-core anyon wavefunctions are

\[ \Psi_{B}(x_{1},x_{2}) = \dfrac{1}{\sqrt{\pi} \alpha^2} e^{-\frac{1}{2\alpha^2} (x_{1}^2 + x_{2}^2)} |x_{2} - x_{1}| \]
\[ \Psi_\kappa(x_{1},x_{2}) = e^{-i \frac{\pi \kappa}{2}  \epsilon(x_{2} - x_{1})} \Psi_{B}(x_{1},x_{2})   \]

where the second relation holds up to a constant phase factor and follows from the anyon-fermi mapping in Section III of this note. So the anyon momentum distribution is

\begin{align*}
n_\kappa (p) & = \dfrac{N}{2 \pi \hbar} \int_{-\infty}^{\infty} dx \int_{-\infty}^{\infty} dx' e^{i \frac{p}{\hbar} (x - x')} \rho_\kappa(x,x')\\
& =  \dfrac{N}{2 \pi \hbar} \int_{-\infty}^{\infty} dx_{2} 
\left|\int_{-\infty}^{\infty} dx  e^{i \frac{p}{\hbar} x} \Psi_\kappa^*(x,x_{2}) \right|^2
\end{align*}

Consider the integral
\begin{align*}
\int_{-\infty}^{\infty} dx \, e^{i \frac{p}{\hbar} x} \Psi_\kappa^*(x_{1},x_{2}) & = 
\int_{-\infty}^{\infty} dx \, e^{i \frac{p}{\hbar} x}  e^{i \frac{\pi \kappa}{2}  \epsilon(x_{2} - x)} \Psi_{B}^*(x, x_{2})\\
& = \int_{-\infty}^{\infty} dx \, e^{i \frac{p}{\hbar} x} \left( \cos\left(\frac{\pi \kappa}{2}\right) + i \epsilon(x_{2} - x) \sin\left(\frac{\pi \kappa}{2} \right) \right) \Psi_{B}^*(x, x_{2})
\end{align*}
Because $\epsilon(x_{2} - x)\Psi_{B}^*(x, x_{2}) = \Psi_{F}(x, x_{2}) $, in the $p \rightarrow \infty$ limit, the $\sin$ term above results in an exponentially decaying term, which means it can be neglected because the $\cos$ term results in a power law $1/p^4$ decay. Thus,

\begin{align*}
\lim_{p \rightarrow \infty} \left[ \int_{-\infty}^{\infty} dx  e^{i \frac{p}{\hbar} x} \Psi_\kappa^*(x_{1},x_{2}) \right] & =  
\cos\left(\frac{\pi \kappa}{2}\right) \int_{-\infty}^{\infty} dx  e^{i \frac{p}{\hbar} x} \Psi_{B}^*(x, x_{2})\\
& = -\cos\left(\frac{\pi \kappa}{2}\right) \dfrac{2 \hbar^2 e^{-(x_{2}/\alpha)^{2}}}{\sqrt{\pi}\alpha^2 p^2}
\end{align*}
where the last equality comes from the integrals evaluated in the previous section and $Q_{2}$ is shorthand for $x_{2}/\alpha$. So,
\begin{align*}
\lim_{p \rightarrow \infty} n_\kappa (p) & = \dfrac{2 N \hbar^3}{\alpha^4 \pi^2 p^4} 
\cos^2\left(\frac{\pi \kappa}{2}\right)  \int_{-\infty}^{\infty} dx_{2}  e^{-2(x_{2}/\alpha)^{2}}\\
& = \dfrac{2}{\pi} \sqrt{\dfrac{2}{\pi}} \cos^2\left(\frac{\pi \kappa}{2}\right) \dfrac{(\hbar m \omega)^{3/2}}{p^4}
\end{align*}

Numerical simulation can confirm the analytic results I have just cited. We can infer the anyon momentum tail coefficient from numerical values for $n_{\kappa}(p)$.  The plot in fig. \ref{fig:AnyonCoeff} compares the analytic and numerical calculation for the tail coefficient divided by $ \frac{2}{\pi} \sqrt{\frac{2}{\pi}}$ (should just be $\cos^2\left(\frac{\pi \kappa}{2}\right)$). This confirms the calculations from Sections IV and V.

If we define the momentum tail coefficient as the Tan contact, then
\[
C_{\kappa} = \cos^2\left(\frac{\pi \kappa}{2}\right) C_{B}
\]
which is consistent with the result obtained in Eur. Phys. J. D \textbf{71}, 135 (2017). Note that in that paper, the anyon statistical parameter $\chi$ is equal to $(1 - \kappa)$ in our notation.

An anyon gas with interaction strength $g$ has the same energy as a bosonic gas with interaction strength $g' = \frac{g}{\cos\left(\frac{\pi \kappa}{2}\right)}$. Therefore, in the large $g$ limit, the two harmonically trapped anyons should have energy
\[
E_{\kappa} = E_{0} - 2\sqrt{\dfrac{2}{\pi}} \frac{1}{g'} 
= E_{0} - 2\sqrt{\dfrac{2}{\pi}} \frac{\cos\left(\frac{\pi \kappa}{2}\right)}{g}
\]
Therefore, it seems that the adiabatic sweep theorem for anyon gas should read
\[
\frac{dE_{\kappa}}{d(1/g)} = - 2 \cos\left(\frac{\pi \kappa}{2}\right) \sqrt{\frac{2}{\pi}} = -\pi C_{\kappa} / \cos\left(\frac{\pi \kappa}{2}\right)
\]

\begin{center}
\begin{figure}[h]
	\includegraphics[scale=.5]{"../Plots/AnyonCoeff"}
	\caption{Anyon momentum tail coefficient plotted as a function of $\kappa$}
	\label{fig:AnyonCoeff}
\end{figure}
\end{center}

\section{Time Evolution of Momentum Tail for expanding hca}

Let's consider the case where two hard-core anyons in the ground state of a harmonic trap are released suddenly. That is, the trap frequency $\omega$ changes suddenly from $\omega_{0}$ to 0. The time evolution of a particle in a harmonic trap with a time dependent trap frequency can be found analytically. 

Let the potential $V(x) = \frac{1}{2} m \omega^2(t) x^2$ where $\omega(t) = \omega_0$ for $t < 0$ and let $\phi_j(x; 0)$ be the $j^{th}$ excited state solution for the static case with trap frequency $\omega_0$. If $\phi_j(x; t)$ is the solution for the time dependent case, then 
\[
\phi_j(x; t) = \frac{1}{\sqrt{b}}\phi_j(x/b; 0) e^{i (\frac{x^2}{2} \frac{\dot{b}}{b} - E_j \tau (t) ) }
\]

where $b(t)$ and $\tau(t)$ are determined by 

\begin{align*}
\ddot{b} + \omega^2(t) b = b^{-3}\\
\tau(t) = \int^{t}_{0}dt' \, b^{-2}(t')
\end{align*}

In the sudden expansion case where $\omega(t) = \omega_0$ for $t < 0$ and $\omega(t) = 0$ for $t > 0$, solving these equations with the initial conditions that $b(0) = 1$ and $b'(0) = 0$ gives
\begin{eqnarray}
b(t) = \sqrt{1 + t^2}\\
\tau(t) = \arctan(t)
\end{eqnarray}

These equations can be used to obtain time dependent expressions for the anyonic wavefunction, one-body density matrix, and momentum distribution.

\begin{align*}
\Psi_{\kappa}(x_{1}, x_{2}; t) & = 
\frac{1}{b} \Psi_{\kappa}(\frac{x_{1}}{b}, \frac{x_{2}}{b}; 0)
\exp \left[ \frac{i \dot{b}}{2 b} (x_{1}^2 + x_{2}^2) - i \tau(t) (E_0 + E_1) \right]\\
\rho_{\kappa}(x, x'; t) & = \frac{1}{b} \rho_{\kappa}(\frac{x}{b}, \frac{x'}{b}; 0)
\exp \left[ \frac{i \dot{b}}{2 b} (x'^2 - x^2) \right]\\
n_{\kappa}(p ; t) & = \dfrac{N}{2\pi b} \int dx \int dx' 
\rho_{\kappa}(\frac{x}{b}, \frac{x'}{b}; 0)
\exp \left[ip(x - x') +	 \frac{i \dot{b}}{2 b} (x'^2 - x^2) \right]
\end{align*}
where
\begin{align*}
\Psi_{\kappa}(x_{1}, x_{2}; 0) & = 
e^{i \pi (1 - \kappa) \theta(x_{2} - x_{1})} \Psi_{F}(x_{1}, x_{2}; 0)\\
\rho_{\kappa}(x, x'; 0) & = \int dx_2 \, \Psi_{\kappa}^*(x,x_2; 0) \Psi_\kappa(x',x_2; 0)
\end{align*}

\begin{center}
\begin{figure}[h!]
	\includegraphics[scale=.43]{"../Plots/MomDistExpandingPair"}
	\caption{Fermionization of HCB (left) and HCA (right $\kappa = \frac{1}{2}$) -- dashed red line corresponds to the fermion distribution}
	\label{fig:HCBFermionizationPair}
\end{figure}
\end{center}

Minguzzi et al. (Phys. Rev. Lett. 94, 240404) show using the stationary phase method that as $t \rightarrow \infty$ the momentum distribution of hard-core bosons approaches the fermion momentum distribution. Their numerical calculations also confirm this. I have performed numerical calculation to reproduce this fermionization for two hard-core bosons. These numerical results are depicted in fig. \ref{fig:HCBFermionizationPair}. 

Using numerical simulation, I also confirmed that two expanding hard-core anyons will experience the same fermionization in their momentum distribution. These results are also in fig. \ref{fig:HCBFermionizationPair} in the plot on the right.

To calculate the time evolution of the momentum tail coefficient, we start by simplifying the expression for the momentum distribution to obtain
\begin{align*}
n_{\kappa}(p ; t) & = \dfrac{N b}{2\pi} \int du \int dv 
\rho_{\kappa}(u, v; 0)
\exp \left[ibp(u - v) +	 \frac{i \dot{b}b}{2 } (v^2 - u^2) \right]\\
& = \dfrac{N b}{2\pi} \int dx_{2} \left| \int dx \, e^{ibpx} \Psi_{\kappa}(x, x_{2}; 0) e^{\frac{i \dot{b}b}{2} x^2}\right|^2
\end{align*}
%
Evaluating the integral in the modulus square we obtain
%
\begin{align*}
\int dx \, e^{ibpx} \Psi_{\kappa}(x, x_{2}; 0) e^{\frac{i \dot{b}b}{2} x^2} 
= \cos\left(\frac{\pi \kappa}{2}\right) 
\frac{1}{\sqrt{\pi}} e^{- x_{2}^2} \frac{-2}{b^2 p^2} e^{i \delta} 
\text{	as } p \rightarrow \infty 
\end{align*}
where $\delta$ is a constant phase shift that will be eliminated by the modulus square and can therefore be ignored. Thus,
\[
\lim_{p \rightarrow \infty} n_{\kappa}(p ; t) = 
\frac{4N}{2 \pi^2 b^3 p^4} \cos^2\left(\frac{\pi \kappa}{2}\right) \sqrt{\frac{\pi}{2}}
= \frac{2}{\pi} \sqrt{\frac{2}{\pi}} \cos^2\left(\frac{\pi \kappa}{2}\right) \frac{1}{b^3 p^4}
\]
\begin{equation}
\lim_{p \rightarrow \infty} n_{\kappa}(p ; t) = 
\frac{2}{\pi} \sqrt{\frac{2}{\pi}} \cos^2\left(\frac{\pi \kappa}{2}\right) \frac{1}{b^3 p^4}
\label{MomentumTail}
\end{equation}

I made two numerical calculations of the time evolution of the momentum tail coefficient for $\kappa = 0$ and $\kappa = \frac{1}{2}$. Both numerical calculations were in agreement with the analytic prediction given in equation 11. Plots of the time evolution are depicted in fig. \ref{fig:HCATimeDepPair}
%
\begin{center}
\begin{figure}[h]
	\includegraphics[scale=.44]{"../Plots/MomTailCoeffTimeDepRow"}
	\caption{Time evolution of momentum tail coefficient for $\kappa = 0$ (left) and similarly for $\kappa = 1/2$ (right)} 
	\label{fig:HCATimeDepPair}
\end{figure}
\end{center}
%
\section{Time Evolution of oscillating hca}

In the last section, we considered the case where the trap frequency of the harmonic trap is suddenly changed from $\omega_0$ to 0. Now, we will consider the case where the trap frequency is changed from $\omega_0$ to a non-zero value $\omega_1$.

Again we can use the same equations as before to describe the time evolution of the system. It can be shown that if $\omega(t) = \omega_0$ for $t \leq 0$ and $\omega(t) = \omega_1$ for $t > 0$, then
\begin{align*}
b(t) = \sqrt{1 + \frac{\omega_0^2 - \omega_1^2}{\omega_1^2} \sin^2(\omega_1 t)}\\
\tau(t) = \frac{1}{\omega_0} \arctan\left(\frac{\omega_0}{\omega_1} \tan(\omega_1 t)\right)
\end{align*}
Under these conditions, the wave function and the momentum distribution will undergo oscillations with period $T = \pi/\omega_1$. Using the same expressions for the OBDM and momentum distribution from the previous section, we can plot the time evolution of the momentum distribution by solving the integrals numerically. 

In fig. \ref{fig:OscHCATimeDepPair}, there are two plots depicting the time evolution of the oscillating anyon gas for the hard-core boson case ($\kappa = 0$) and the $\kappa = 1/2$ case. For the sake of clarity, only half the period is depicted (where full period $T = 3\pi$) because the time evolution of the latter half of the period is the same as the first half but reversed. So, the plots in fig. \ref{fig:OscHCATimeDepPair} only give time slices from $t = 0$ to $t = \frac{3\pi}{2}$ 

The expression for the momentum tail coefficient in equation \ref{MomentumTail} can be used to find the momentum tail coefficient in the oscillating case, as long as the appropriate function is used for $b(t)$. In fig. \ref{fig:MomTailCoeffComp} there is a plot comparing the time evolution of the momentum tail coefficient for the expansion case and the oscillating case. The time evolution of each has been plotted over one full period where $T = 3\pi$. The momentum tail coefficient decays to zero in the expanding gas case, but oscillates periodically in the case where the the trap frequency is changed to a non-zero value.

\begin{center}
\begin{figure}[h]
	\includegraphics[scale=.52]{"../Plots/MomDistOscillatingPair"}
	\caption{Time evolution of oscillating anyon gas $\kappa = 0$ (left) and similarly for $\kappa = 1/2$ (right)} 
	\label{fig:OscHCATimeDepPair}
\end{figure}
\end{center}

\begin{center}
\begin{figure}[h]
	\includegraphics[scale=.4]{"../Plots/MomTailCoeffComparison"}
	\caption{Time evolution of momentum tail in expanding case and oscillating case} 
	\label{fig:MomTailCoeffComp}
\end{figure}
\end{center}

\section{momentum tail coefficient for hcb (n-particle case)}

In this section I will show that for $N$ hard-core bosons in a harmonic trap, the momentum distribution has a momentum tail that decays like $1/p^4$. It can be shown that (see Girardeau et al. \footnotemark\,)
\footnotetext{M. D. Girardeau, E. M. Wright, and J. M. Triscari, Physical Review A \textbf{63} 033601 }
\begin{align*}
\Psi_B(x_{1}, x_{2}, \ldots, x_{N}) & = A_{N} \exp\left(-\frac{1}{2} \sum_{i = 1}^{N} x_{i}^2\right) \prod_{1 \leq j < k \leq N} |x_{k} - x_{j}|\\
\rho(x, x') & = \int dx_{2} \cdots \int dx_{N} \Psi_{B}^*(x, x_{2}, \ldots, x_{N}) \Psi_{B}(x', x_{2}, \ldots, x_{N})
\end{align*}
where $A_{N} = \dfrac{2^{(N - 1)N/4}}{\pi^{N/4}} \left( \prod_{n = 0}^{N} n! \right)^{-1/2}$. Thus, by letting $\tilde{A}_{N} = \frac{N}{2 \pi} A_{N}^2$
%
\begin{align*}
n_{B}(p) &= \dfrac{N}{2 \pi} \int dx \int dx' e^{i p (x - x')} \rho(x, x')\\
& = \dfrac{N}{2 \pi} \int dx_{2} \cdots \int dx_{N} \left| \int dx e^{i p x} \Psi_{B}^*(x, x_{2}, \ldots, x_{N}) \right|^2\\
& = \tilde{A}_{N} \int dx_{2} \cdots \int dx_{N} \exp\left(- \sum_{i = 2}^{N} x_{i}^2\right) \left( \prod_{2 \leq i < q \leq N} (x_{q} - x_{i})^2 \right)
	\left| \int dx e^{i p x} e^{-\frac{1}{2} x^2} \prod_{2 \leq j \leq N} |x_{j} - x| \right|^2
\end{align*} 
%
For large $p$, we can approximate the integral inside the modulus as
\[
\int dx e^{i p x} e^{-\frac{1}{2} x^2} \prod_{2 \leq j \leq N} |x_{j} - x| = \left( \dfrac{-2}{p^2} \right) \sum_{j = 2}^{N} e^{-\frac{1}{2} x_{j}^2} e^{i p x_{j}} \prod_{k \neq j} |x_{k} - x_{j}|
\]
Plug this back into the original expression and abbreviating the condition under the product symbol
\[
n_{B}(p) = \tilde{A}_{N} \frac{4}{p^4} \int dx_{2} \cdots \int dx_{N} \exp\left(- \sum_{i = 2}^{N} x_{i}^2\right) \prod_{ i < q} (x_{q} - x_{i})^2 
	\sum_{j = 2}^{N} \sum_{\ell = 2}^{N} e^{-\frac{1}{2} (x_{j}^2 + x_{\ell}^2)} e^{i p (x_{j} - x_{\ell})} 
	\prod_{k \neq j} |x_{k} - x_{j}| \prod_{m \neq \ell} |x_{m} - x_{\ell}|
\]
Now, whenever $x_{j} \neq x_{\ell}$ in the double sum, there will be $e^{i p (x_{j} - x_{\ell})}$ term. Elements in the double sum with this term (after they are integrated over) will lead to higher order dependencies (e.g., $1/p^2$), and once these terms are multiplied by $1/p^4$ in the prefactor, they will no longer be leading order. Thus, we can neglect any terms in the double sum where $x_{j} \neq x_{\ell}$, giving
\begin{align*}
n_{B}(p) & = \tilde{A}_{N} \frac{4}{p^4} \int dx_{2} \cdots \int dx_{N} \exp\left(- \sum_{i = 2}^{N} x_{i}^2\right) \prod_{ i < q} (x_{q} - x_{i})^2 
	\sum_{j = 2}^{N} e^{-x_{j}^2} \prod_{k \neq j} (x_{k} - x_{j})^2\\
	& = \tilde{A}_{N} \frac{4}{p^4} \sum_{j = 2}^{N} \int dx_{2} \cdots \int dx_{N} 
	\exp\left(- \sum_{i = 2}^{N} x_{i}^2\right) \left( \prod_{ i < q} (x_{q} - x_{i})^2 \right) 
	e^{-x_{j}^2} \prod_{k \neq j} (x_{k} - x_{j})^2\\
\end{align*}

Because each integral in this sum is the same but with the indices rotated, we can write
\begin{align*}
n_{B}(p) & = \tilde{A}_{N} \frac{4 * (N - 1)}{p^4} \int dx_{2} \cdots \int dx_{N} 
	\exp\left(- \sum_{i = 2}^{N} x_{i}^2\right) \left( \prod_{ i < q} (x_{q} - x_{i})^2 \right) 
	e^{-x_{2}^2} \prod_{k \neq 2} (x_{k} - x_{2})^2\\
	& = \dfrac{C_{N}}{p^4}\\
C_{N} & = 4(N - 1)\tilde{A}_{N}  \int dx_{2} \cdots \int dx_{N} 
	\exp\left(- \sum_{i = 2}^{N} x_{i}^2\right) \left( \prod_{ i < q} (x_{q} - x_{i})^2 \right) 
	e^{-x_{2}^2} \prod_{k \neq 2} (x_{k} - x_{2})^2\\
\end{align*}
I have tried to simplify the expression for $A_{N}$ as much as I can, but I still have not been able to find an analytic expression for it. Nonetheless, this still shows that the momentum distribution has a $1/p^4$ momentum decay for arbitrary $N$. For numerical confirmation of this result, see next section

\section{momentum tail coefficient for hca (n-particle case)}

In this section, I will find the momentum tail for $N$ hard-core anyons. We will be able to use some of the results from the previous section to help us. The anyon-fermi mapping is given by
\[
A_{\kappa}(x_{1},x_{2}, \ldots, x_{N}) = \prod_{2 \leq j < k \leq N} e^{i\pi(1 - \kappa) \theta(x_{k} - x_{j})} = \exp\left[ i\pi(1-\kappa) \left( \sum_{j < k} \theta(x_{k} - x_{j}) \right) \right]
\]
Thus, the anyon wave function is 
\[
\Psi_{\kappa}(x_{1}, \ldots, x_{N}) = A_{N} \times A_{\kappa}(x_{1},\ldots, x_{N}) \times \exp\left(-\frac{1}{2} 
\sum_{i = 1}^{N} x_{i}^2\right) \prod_{1 \leq j < k \leq N} (x_{k} - x_{j})
\]
But as was mentioned in a previous section, we can also write this as
\[
\Psi_{\kappa}(x_{1}, \ldots, x_{N}) = A_{N}  \exp\left(-\frac{1}{2} 
\sum_{i = 1}^{N} x_{i}^2\right) \exp\left( i\frac{\pi \kappa}{2} \Sigma_{j < k} \epsilon(x_{k} - x_{j}) \right) \prod_{1 \leq j < k \leq N} |x_{k} - x_{j}|
\]
Letting $\epsilon_{\kappa} \equiv \exp(i\frac{\pi \kappa}{2} \Sigma_{j=2}^N \epsilon(x_{j} - x))$ and $\tilde{A}_{N} = \frac{N}{2 \pi} A_{N}^2$, then the momentum distribution is
\begin{align*}
n_{\kappa}(p) & = \frac{N}{2\pi} \int dx_{2} \cdots \int dx_{N} \left| \int dx e^{ipx} \Psi_{\kappa}^*(x, x_{2}, \ldots, x_{N}) \right|^2\\
& = \frac{N}{2\pi} \int dx_{2} \cdots \int dx_{N} \left| \int dx e^{ipx} \exp(i\frac{\pi \kappa}{2} \Sigma_{j=2}^N \epsilon(x_{j} - x)) \Psi_{B}^*(x, x_{2}, \ldots, x_{N}) \right|^2\\
& = \tilde{A}_{N} \int dx_{2} \cdots \int dx_{N} \exp\left(- \sum_{i = 2}^{N} x_{i}^2\right) \left( \prod_{2 \leq i < q \leq N} (x_{q} - x_{i})^2 \right)
	\left| \int dx e^{i p x} e^{-\frac{1}{2} x^2} \epsilon_{\kappa} \prod_{2 \leq j \leq N} |x_{j} - x| \right|^2
\end{align*}
Expanding the $\epsilon_{\kappa}$ in the modulus square we obtain
\begin{align*}
\epsilon_{\kappa} & = e^{i\frac{\pi \kappa}{2}\epsilon(x_{2} - x)} \cdots e^{i\frac{\pi \kappa}{2}\epsilon(x_{N} - x)}\\
& = \left(\cos(\pi \kappa/2) + i\epsilon(x_{2} - x)\sin(\pi \kappa/2) \right) 
\cdots \left(\cos(\pi \kappa/2) + i\epsilon(x_{N} - x)\sin(\pi \kappa/2)\right)
\end{align*}
To show how we should handle this product, it will be easier to look at the case for $N = 3$, where
\begin{align*}
 \int dx e^{ipx} \left(\cos(\pi \kappa/2) + i\epsilon(x_{2} - x)\sin(\pi \kappa/2)\right) 
\left(\cos(\pi \kappa/2) + i\epsilon(x_{3} - x)\sin(\pi \kappa/2)\right) e^{-\frac{1}{2} x^2}  \prod_{2 \leq j \leq 3} |x_{j} - x| \\
= \int dx e^{ipx} \left( \cos^2(\pi \kappa/2) 
+ i\frac{1}{2}\epsilon(x_{2} - x)\sin(\pi \kappa) + i\frac{1}{2}\epsilon(x_{3} - x)\sin(\pi \kappa)
 - \epsilon(x_{2} - x)\epsilon(x_{3} - x) \sin^2(\pi \kappa/2) \right) e^{-\frac{1}{2} x^2}  \prod_{2 \leq j \leq 3} |x_{j} - x|
\end{align*}
%
The integral over the first term can be found easily by using the calculations from the boson case. I will show that all other terms that have a $\epsilon$ function in them will not be leading order and can therefore be ignored. For instance, let's look at the integral over $x$ for the second term (ignoring prefactors)
%
\begin{align*}
\int dx e^{i p x} e^{-\frac{1}{2} x^2} \epsilon(x_{2} - x) |x_{2} - x| |x_{3} - x|
&= \int dx e^{i p x} e^{-\frac{1}{2} x_{2}^2} \epsilon(x_{2} - x) |x_{2} - x| |x_{3} - x_{2}| \\
& + \int dx e^{i p x} e^{-\frac{1}{2} x_{3}^2} \epsilon(x_{2} -x_{3}) |x_{2} - x_{3}| |x_{3} - x| + o (\frac{1}{p^2})\\
&= 0 - \frac{2}{p^2} e^{ipx_{3}}e^{-\frac{1}{2} x_{3}^2} \epsilon(x_{2} -x_{3}) |x_{2} - x_{3}| 
\end{align*}

Using \textit{Applications of Fourier Transforms to Generalized Functions} by M. Rahman (see the table on p. 159) to evaluate these integrals. The first integral is zero because the fourier transform of $\epsilon(x_{2} - x) |x_{2} - x|$ with respect to $x$ is 0. Evaluating the integral with respect to $x$ over other terms in the sum yields
\begin{align*}
\int dx e^{i p x} e^{-\frac{1}{2} x^2} \epsilon_{\kappa} \prod_{2 \leq j \leq 3} |x_{j} - x| = \left( \frac{-2}{p^2} \right) 
 \cos^2(\frac{\pi \kappa}{2}) \sum_{j=2}^{3} e^{-x_{j}^2/2 + ipx_{j}} |x_{3} - x_{2}|\\
 + \frac{i}{2} \sin(\pi \kappa) \left( \frac{-2}{p^2} \right) |x_{2} - x_{3}| \epsilon(x_{2} - x_{3})
 \sum_{j=2}^{3} (-1)^{j} e^{-x_{j}^{2}/2 + ipx_{j}} 
\end{align*}
where the last term in the sum with two $\epsilon$ functions will evaluate to zero. Now, to find the momentum distribution for the $N = 3$ case, we must find the modulus square of this expression. I will leave out the derivation for the following result, but if you use the same reasoning for the boson case you will find that 

\begin{align*}
\left| \int dx e^{i p x} e^{-\frac{1}{2} x^2} \epsilon_{\kappa} \prod_{2 \leq j \leq 3} |x_{j} - x| \right|^2 & = 
\frac{4}{p^4} |x_{2}-x_{3}|^2 \left[ \cos^4(\pi \kappa /2) \sum_{j} e^{-x_{j^2}} + \frac{\sin^2(\pi \kappa)}{4} \epsilon(x_{2} - x_{3})^2 \sum_{j} e^{-x_{j}^2} \right]\\
& = \frac{4}{p^4} |x_{2}-x_{3}|^2 \sum_{j} e^{-x_{j^2}} \left[ \cos^4(\pi \kappa /2) + \frac{\sin^2(\pi \kappa)}{4} \right]\\
& = \cos^2(\pi \kappa /2)\frac{4}{p^4} |x_{2}-x_{3}|^2 \sum_{j} e^{-x_{j^2}}
\end{align*}
Comparing this with the $N$-particle expression for bosons, we can see that for the $3$ particle case that for large $p$
\[
n_{\kappa}(p) = \cos^2(\pi \kappa /2) \frac{C_{3}}{p^4}
\]

It is not clear that for the $N$-particle case whether the $\kappa$ dependence can be expressed as the power of a cosine function. Letting $\epsilon_{ij} \equiv \epsilon(x_{i} - x_{j}) $ and similarly $|x_{ij}| \equiv |x_{i} - x_{j}|$ recall that
%
\begin{align*}
n_{\kappa}(p) & = \tilde{A}_{N} \int dx_{2} \cdots \int dx_{N} \exp\left(- \sum_{i = 2}^{N} x_{i}^2\right) \left( \prod_{2 \leq i < q \leq N} (x_{q} - x_{i})^2 \right)
	\left| \int dx e^{i p x} e^{-\frac{1}{2} x^2} \epsilon_{\kappa} \prod_{2 \leq j \leq N} |x_{j} - x| \right|^2\\
\epsilon_{\kappa} & =  \left(\cos(\pi \kappa/2) + i\epsilon(x_{2} - x)\sin(\pi \kappa/2) \right) 
\cdots \left(\cos(\pi \kappa/2) + i\epsilon(x_{N} - x)\sin(\pi \kappa/2)\right)\\
& = \sum_{q = 1}^{N} \cos^{N-q}(\pi \kappa /2) (i \sin(\pi \kappa /2))^{q-1}
\sum_{2\leq i_{1} < \ldots < i_{q - 1} \leq N} \epsilon(x_{i_{1}} - x) \cdots \epsilon(x_{i_{q-1}} - x)
\end{align*}

So making extensive use of Theorem 4.4 from Rahman, and abbreviating the summation condition for $i_{1}, i_{2}, \ldots, i_{q - 1}$
\begin{align*}
\int & dx e^{i p x} e^{-\frac{1}{2} x^2} \epsilon_{\kappa} \prod_{2 \leq j \leq N} |x_{j} - x| =\\
-\frac{2}{p^2}& \left[ \sum_{q = 1}^{N} \cos^{N-q}(\pi \kappa /2) (i \sin(\pi \kappa /2))^{q-1}
 \sum_{i_{1} < \ldots < i_{q - 1}} \sum_{j \neq i_{1}, \ldots, i_{q-1}} e^{-x_{j}^2/2 + ipx_{j}} \epsilon_{i_{1}j} \cdots \epsilon_{i_{q-1}j} \prod_{k \neq j} |x_{kj}| \right]
\end{align*}

The modulus square of this expression is then
\begin{align*}
 & \left| \int dx e^{i p x}  e^{-\frac{1}{2} x^2} \epsilon_{\kappa} \prod_{2 \leq j \leq N} |x_{j} - x| \right|^2 =\\
\frac{4}{p^4}  \sum_{q = 1}^{N} & \sum_{r = 1}^{N} \cos(\pi \kappa /2)^{2N-q-r} (i \sin(\pi \kappa /2))^{r+q-2} (-1)^{r-1} 
\sum_{i_{1} < \ldots < i_{q - 1}} \sum_{j_{1} < \ldots < j_{r - 1}}\\
&\sum_{k \neq i} 
\sum_{\ell \neq j} 
e^{-(x_{k}^{2} + x_{\ell}^{2})/2} e^{ip(x_{k} - x_{\ell})} 
\epsilon_{i_{1}k} \cdots \epsilon_{i_{q-1}k}
\epsilon_{j_{1}\ell} \cdots \epsilon_{j_{r-1}\ell}
\prod_{\alpha \neq k} |x_{\alpha k}|
\prod_{\alpha \neq \ell} |x_{\alpha \ell}|
\end{align*}
where the sums are all supposed to be in one line but I have split them into two to fit them on the page. Suppose that $k \neq \ell$. Then the integral with respect to $x_{k}$ will be a fourier transform. This can only lead to vanishing terms or non-leading order terms. Due to all the singularities, this fourier transform will be split into a sum over every singularity as per Theorem 4.4. If every fourier transform in this resultant sum is negligible, then the fourier transform overall is negligible. Here I have enumerated every possible fourier transform that could result

\begin{enumerate}
\item $\mathcal{F}[|x_{k} - x_{i}|] = \mathcal{O}(\frac{1}{p^2})$
\item $\mathcal{F}[\epsilon(x_{k} - x_{i})|x_{k} - x_{i}|] = 0$
\item $\mathcal{F}[\epsilon(x_{k} - x_{i})|x_{k} - x_{i}|^2] = \mathcal{O}(\frac{1}{p^3})$
\item $\mathcal{F}[\epsilon(x_{k} - x_{i})^2|x_{k} - x_{i}|^2] = \mathcal{O}(\frac{1}{p^3})$
\item $\mathcal{F}[|x_{k} - x_{i}|^2] = \mathcal{O}(\frac{1}{p^3})$
\end{enumerate}
When any of these terms is multiplied by the $\frac{4}{p^4}$ prefactor, that term will no longer be leading order, and it can therefore be ignored. So, we can ignore any terms with $k \neq \ell$ giving us

\begin{align*}
\frac{4}{p^4}  \sum_{q = 1}^{N} & \sum_{r = 1}^{N} \cos(\pi \kappa /2)^{2N-q-r} (i \sin(\pi \kappa /2))^{r+q-2} (-1)^{r-1} 
\sum_{i_{1} < \ldots < i_{q - 1}} \sum_{j_{1} < \ldots < j_{r - 1}}\\
&\sum_{k \neq i, j} 
e^{-x_{k}^{2}} 
\epsilon_{i_{1}k} \cdots \epsilon_{i_{q-1}k}
\epsilon_{j_{1}k} \cdots \epsilon_{j_{r-1}k}
\prod_{\alpha \neq k} |x_{\alpha k}|^2
\end{align*}

%\section{two anyons in harmonic trap with finite contact interaction}
%Consider the Hamiltonian for two anyons:
%\[ H = -\frac{1}{2} \left( \partial^2_{x_1} +\partial^2_{x_2} \right) + \frac{1}{2}(x_1^2+x_2^2) + g\delta(x_1-x_2)  \] This can be decomposed to the COM and relative coordinates. The COM Hamiltonian is just the harmoinc oscillator, while the relative Hamiltonian takes the form ($x\equiv x_1-x_2$): \[ H_{\rm rel} = -\frac{1}{2} \partial^2_x + \frac{1}{2} x^2 + g\delta(x) \] The relative wavefunction has to satisfiy \[  \phi(x) = e^{i\pi \kappa \epsilon(x)} \,\phi(-x) \] This constraint, however, seems to indicate that the wavefunction is discontinuous at $x=0$. If this is the case, then the second order derivative term will result in a derivative in delta function, which cannot be cancelled out. The paper PRL {\bf 83}, 1275 (1998) seems to be relevant, but I am not sure how they got from Eq. (16) to (17).
	

\end{document}
